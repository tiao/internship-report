% ----------------------------------------------------------------------- %
% Arquivo: conclusoes.tex
% ----------------------------------------------------------------------- %

\chapter{Conclus�es e Perspectivas}


Este trabalho mostrou-me a import�ncia da experi�ncia profissional para aluno, pois e o
momento de por em pr�tica os conhecimentos adquiridos na gradua��o. Pois, muitas vezes
somente o conhecimento acad�mico n�o e o suficiente para a solu��o dos problemas. Foram
in�meros os conhecimentos que adquiri no per�odo de trabalho, mas alguns defino como os
mais importantes, porque hoje s�o um diferencial no mercado de trabalho:

\begin{itemize}
 \item gest�o de projetos;
 \item desenvolvimentos nas linguagens C e Assembly;
 \item utilizac�o de equipamentos de metrologia, para auxiliar na resolu��o de problemas;
 \item projeto e desenvolvimento de placas de circuito impresso, que necessitam muita familiaridade
 \item com datasheets de componentes, seus encapsulamentos e compatibilidade el�trica;
\end{itemize}

Com a experi�ncia do trabalho pude constatar o quanto s�o dif�ceis projetos que envolvem
integra��o de software e hardware, se algumas metologias n�o forem aplicadas torna-se quase
imposs�vel alcan�ar algum exito. Outro ponto importante que a experi�ncia de trabalho
 proporciona, e a intera��o com as pessoas, n�o somente agregando informa��es t�cnicas como
o desenvolvimento das rela��es interpessoais.

Como sugest�o ao curso acho interessante um maior aprofundamento em t�cnicas de programa��o,
 projeto e desenvolvimento de sistemas embarcados, integra��o software e hardware, enfase em
linguagens de descri��o de hardware.
Porque hoje o mercado demanda muito por profissionais com esses conhecimentos. No entanto,
 o Curso Superior de Tecnologia em Sistemas de Telecomunica��es apresenta tecnologias
atuais e abrangentes, que possibilita atua��o em v�rias �reas, e foi de extrema import�ncia em
minha vida profissional.

